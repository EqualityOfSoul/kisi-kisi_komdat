\documentclass[oneside]{book}
\usepackage{xcolor}


\newcommand{\exercisename}{Latihan}
\newcommand{\solutionname}{Solusi}

\definecolor{main}{RGB}{0,120,2}

%% Exercise with counter
\newcounter{exer}[chapter]
\setcounter{exer}{0}
\renewcommand{\theexer}{\thechapter.\arabic{exer}}
\newenvironment{exercise}[1][]{
  \refstepcounter{exer}
  \par\noindent\textbf{\color{main}{\exercisename} \theexer #1 }\rmfamily}{
  \par\ignorespacesafterend}

\newenvironment{solution}{\par\noindent\textbf{\color{main}\solutionname} \em}{\par}

\begin{document}


\chapter{Introduction}

% Pritami Sergio 1-4
\begin{exercise}
    Contoh soal 1
\end{exercise}

\begin{solution}
    Contoh solusi
\end{solution}

\vspace{12pt}

\begin{exercise}
  Contoh soal
\end{exercise}

\begin{solution}
  Contoh solusi
\end{solution}

\vspace{12pt}

\begin{exercise}
  Contoh soal
\end{exercise}

\begin{solution}
  Contoh solusi
\end{solution}

\vspace{12pt}

\begin{exercise}
  Contoh soal
\end{exercise}

\begin{solution}
  Contoh solusi
\end{solution}

\vspace{12pt}

% Muhammad Adi Syahputra 5-8
\begin{exercise}
  Contoh soal 5
\end{exercise}

\begin{solution}
  Contoh solusi
\end{solution}

\vspace{12pt}

\begin{exercise}
  Contoh soal
\end{exercise}

\begin{solution}
  Contoh solusi
\end{solution}

\vspace{12pt}

\begin{exercise}
  Contoh soal
\end{exercise}

\begin{solution}
  Contoh solusi
\end{solution}

\vspace{12pt}

\begin{exercise}
  Contoh soal
\end{exercise}

\begin{solution}
  Contoh solusi
\end{solution}

\vspace{12pt}

% Amelia Marta Dilova 9-12
\begin{exercise}
  Contoh soal 9
\end{exercise}

\begin{solution}
  Contoh solusi
\end{solution}

\vspace{12pt}

\begin{exercise}
  Contoh soal
\end{exercise}

\begin{solution}
  Contoh solusi
\end{solution}

\vspace{12pt}

\begin{exercise}
  Contoh soal
\end{exercise}

\begin{solution}
  Contoh solusi
\end{solution}

\vspace{12pt}

\begin{exercise}
  Contoh soal 12
\end{exercise}

\begin{solution}
  Contoh solusi
\end{solution}

\chapter{Network Model}

% Muhammad Riyadhtul Akbar 1-3
\begin{exercise}
  Contoh soal 1
\end{exercise}

\begin{solution}
  Contoh solusi
\end{solution}

\vspace{12pt}

\begin{exercise}
  Contoh soal
\end{exercise}

\begin{solution}
  Contoh solusi
\end{solution}

\vspace{12pt}

\begin{exercise}
  Contoh soal
\end{exercise}

\begin{solution}
  Contoh solusi
\end{solution}

\vspace{12pt}

% Muhammad Rizky Fadillah 4-7
\begin{exercise}
  Contoh soal 4
\end{exercise}

\begin{solution}
  Contoh solusi
\end{solution}

\vspace{12pt}

\begin{exercise}
  Contoh soal
\end{exercise}

\begin{solution}
  Contoh solusi
\end{solution}

\vspace{12pt}

\begin{exercise}
  Contoh soal
\end{exercise}

\begin{solution}
  Contoh solusi
\end{solution}

\vspace{12pt}

\begin{exercise}
  Contoh soal
\end{exercise}

\begin{solution}
  Contoh solusi
\end{solution}

\vspace{12pt}

% Rizky Sandiary 8-11
\begin{exercise}
\\
Bagaimana OSI dan ISO terkait satu sama lain?
\end{exercise}

\begin{solution}
ISO adalah organisasi (Organisasi Standar Internasional), dan OSI (Interkoneksi Sistem Terbuka) adalah modelnya.
\end{solution}

\vspace{12pt}

\begin{exercise}
 \\
Misalkan komputer mengirimkan paket pada lapisan jaringan ke komputer lain di suatu tempat di Internet. Alamat tujuan logis dari paket rusak. Apa yang terjadi pada paket? Bagaimana komputer sumber dapat mengetahui situasinya?
\end{exercise}

\begin{solution}
\\
Sebelum menggunakan alamat tujuan di perantara atau node tujuan, paket melewati pemeriksaan kesalahan yang dapat membantu node menemukan korupsi (dengan probabilitas tinggi) dan membuang paket. Biasanya protokol lapisan atas akan menginformasikan sumber untuk mengirim ulang paket.
\end{solution}

\vspace{12pt}

\begin{exercise}
Jika lapisan data link dapat mendeteksi kesalahan antar hop, mengapa menurut Anda kita memerlukan mekanisme pemeriksaan lain di lapisan transport?
\end{exercise}

\begin{solution}
Kesalahan antar node dapat dideteksi oleh kontrol lapisan data link, tetapi kesalahan pada node (antara port input dan port output) dari node tidak dapat dideteksi oleh lapisan data link
\end{solution}

\vspace{12pt}

\begin{exercise}
Misalkan sebuah komputer mengirimkan sebuah frame ke komputer lain pada topologi bus LAN. Alamat tujuan fisik frame rusak selama transmisi. Apa yang terjadi pada bingkai? Bagaimana pengirim dapat diberitahu tentang situasinya?
\end{exercise}

\begin{solution}
Jika alamat tujuan yang rusak tidak cocok dengan alamat stasiun mana pun di jaringan, paket akan hilang. Jika alamat tujuan yang rusak cocok dengan salah satu stasiun, frame dikirimkan ke stasiun yang salah. Namun, dalam kasus ini, mekanisme pendeteksian kesalahan, yang tersedia di sebagian besar protokol tautan data, akan menemukan kesalahan dan membuang bingkai. Dalam kedua kasus, sumber entah bagaimana akan diinformasikan menggunakan salah satu mekanisme kontrol tautan data.
\end{solution}

\chapter{Data and Signals}

% Dimas Yediberto Luciano Dien 1-3
\begin{exercise}
  Contoh soal 1
\end{exercise}

\begin{solution}
  Contoh solusi
\end{solution}

\vspace{12pt}

\begin{exercise}
  Contoh soal
\end{exercise}

\begin{solution}
  Contoh solusi
\end{solution}

\vspace{12pt}

\begin{exercise}
  Contoh soal
\end{exercise}

\begin{solution}
  Contoh solusi
\end{solution}

\vspace{12pt}

% Karel Chavez H 4-6
\begin{exercise}
  Contoh soal 4
\end{exercise}

\begin{solution}
  Contoh solusi
\end{solution}

\vspace{12pt}

\begin{exercise}
  Contoh soal
\end{exercise}

\begin{solution}
  Contoh solusi
\end{solution}

\vspace{12pt}

\begin{exercise}
  Contoh soal
\end{exercise}

\begin{solution}
  Contoh solusi
\end{solution}

\vspace{12pt}

% Muhammad Arie Setya Putra Pala 7-9
\begin{exercise}
\\
Apa hubungan teorema Nyquist dengan komunikasi?
\end{exercise}

\begin{solution}
\\
Teorema Nyquist-Shannon juga dikenal sebagai teorema pengambilan sampel adalah ketentuan fisik mendasar untuk komunikasi di mana sinyal kontinu dalam waktu terkait dengan sinyal diskrit dalam waktu. Ini pada dasarnya menetapkan jumlah pengambilan sampel minimum yang memungkinkan urutan diskrit untuk menangkap semua sinyal kontinu.
\end{solution}

\vspace{12pt}

\begin{exercise}
\\
Apa hubungan kapasitas Shannon dengan komunikasi?
\end{exercise}

\begin{solution}
\\
Batas Shannon atau kapasitas Shannon dari saluran komunikasi mengacu pada tingkat maksimum data bebas kesalahan yang secara teoritis dapat ditransfer melalui saluran jika tautan mengalami kesalahan transmisi data acak, untuk tingkat kebisingan tertentu.
\end{solution}

\vspace{12pt}

\begin{exercise}
\\
Mengapa sinyal optik yang digunakan pada kabel serat optik memiliki panjang gelombang yang sangat pendek?
\end{exercise}

\begin{solution}
Optical signals have very high frequencies. A high frequency means a short wave length because the wave length is inversely proportional to the frequency.
\end{solution}

\vspace{12pt}

% Ricky 10-12
\begin{exercise}
\\
Bisakah kita mengatakan jika suatu sinyal periodik atau nonperiodik hanya dengan melihat frekuensinya petak domain ? bagaimana ?
\end{exercise}

\begin{solution}
\\
  bisa, karena sinyal periodik dapat dilihat dari frekuensinya yang memiliki periode waktu dasar berulang pada interval waktu yang teratur sedangkan sinyal non-periodik itu acak dan tidak dapat di definisi seperti pada gelombang sinus atau gelombang kosinus.
\end{solution}

\vspace{12pt}

\begin{exercise}
\\
Apakah plot domain frekuensi dari sinyal suara itu diskrit atau kontinu?
\end{exercise}

\begin{solution}
\\
Domain frekuensi sinyal suara biasanya kontinu karena suara adalah sinyal nonperiodik.
\end{solution}

\vspace{12pt}

\begin{exercise}
\\
Apakah plot domain frekuensi dari sistem alarm itu diskrit atau kontinu?
\end{exercise}

\begin{solution}
\\
Sistem alarm biasanya periodik. Oleh karena itu, plot domain frekuensinya adalah diskrit.
\end{solution}

\vspace{12pt}

% Fajar Bimantara 13-16
\begin{exercise}
  Contoh soal 13
\end{exercise}

\begin{solution}
  Contoh solusi
\end{solution}

\vspace{12pt}

\begin{exercise}
  Contoh soal
\end{exercise}

\begin{solution}
  Contoh solusi
\end{solution}

\vspace{12pt}

\begin{exercise}
  Contoh soal
\end{exercise}

\begin{solution}
  Contoh solusi
\end{solution}

\vspace{12pt}

\begin{exercise}
  Contoh soal
\end{exercise}

\begin{solution}
  Contoh solusi
\end{solution}

\vspace{12pt}

% Julicko Pratama Putra 17-19
\begin{exercise}
  Contoh soal 17
\end{exercise}

\begin{solution}
  Contoh solusi
\end{solution}

\vspace{12pt}

\begin{exercise}
  Contoh soal
\end{exercise}

\begin{solution}
  Contoh solusi
\end{solution}

\vspace{12pt}

\begin{exercise}
  Contoh soal
\end{exercise}

\begin{solution}
  Contoh solusi
\end{solution}

\vspace{12pt}

% Nadjamudin Beda 20-23
\begin{exercise}
  Contoh soal 20
\end{exercise}

\begin{solution}
  Contoh solusi
\end{solution}

\vspace{12pt}

\begin{exercise}
  Contoh soal
\end{exercise}

\begin{solution}
  Contoh solusi
\end{solution}

\vspace{12pt}

\begin{exercise}
  Contoh soal
\end{exercise}

\begin{solution}
  Contoh solusi
\end{solution}

\vspace{12pt}

\begin{exercise}
  Contoh soal
\end{exercise}

\begin{solution}
  Contoh solusi
\end{solution}

\chapter{Digital Transmission}

% Pritami Sergio 1-4
\begin{exercise}
  Contoh soal 1
\end{exercise}

\begin{solution}
  Contoh solusi
\end{solution}

\vspace{12pt}

\begin{exercise}
  Contoh soal
\end{exercise}

\begin{solution}
  Contoh solusi
\end{solution}

\vspace{12pt}

\begin{exercise}
  Contoh soal
\end{exercise}

\begin{solution}
  Contoh solusi
\end{solution}

\vspace{12pt}


\begin{exercise}
  Contoh soal
\end{exercise}

\begin{solution}
  Contoh solusi
\end{solution}

\vspace{12pt}

% Muhammad Adi Syahputra 5-8
\begin{exercise}
  Contoh soal 5
\end{exercise}

\begin{solution}
  Contoh solusi
\end{solution}

\vspace{12pt}

\begin{exercise}
  Contoh soal
\end{exercise}

\begin{solution}
  Contoh solusi
\end{solution}

\vspace{12pt}

\begin{exercise}
  Contoh soal
\end{exercise}

\begin{solution}
  Contoh solusi
\end{solution}

\vspace{12pt}

\begin{exercise}
  Contoh soal
\end{exercise}

\begin{solution}
  Contoh solusi
\end{solution}

\vspace{12pt}

% Amelia Marta Dilova 9-12
\begin{exercise}
  Contoh soal 9
\end{exercise}

\begin{solution}
  Contoh solusi
\end{solution}

\vspace{12pt}

\begin{exercise}
  Contoh soal
\end{exercise}

\begin{solution}
  Contoh solusi
\end{solution}

\vspace{12pt}

\begin{exercise}
  Contoh soal
\end{exercise}

\begin{solution}
  Contoh solusi
\end{solution}

\vspace{12pt}

\begin{exercise}
  Contoh soal
\end{exercise}

\begin{solution}
  Contoh solusi
\end{solution}

\vspace{12pt}

% Muhammad Riyadhtul Akbar 13-16
\begin{exercise}
  Contoh soal 13
\end{exercise}

\begin{solution}
  Contoh solusi
\end{solution}

\vspace{12pt}

\begin{exercise}
  Contoh soal
\end{exercise}

\begin{solution}
  Contoh solusi
\end{solution}

\vspace{12pt}

\begin{exercise}
  Contoh soal
\end{exercise}

\begin{solution}
  Contoh solusi
\end{solution}

\vspace{12pt}

\begin{exercise}
  Contoh soal
\end{exercise}

\begin{solution}
  Contoh solusi
\end{solution}

\vspace{12pt}

% Muhammad Rizky Fadillah 17-20
\begin{exercise}
  Contoh soal 17
\end{exercise}

\begin{solution}
  Contoh solusi
\end{solution}

\vspace{12pt}

\begin{exercise}
  Contoh soal
\end{exercise}

\begin{solution}
  Contoh solusi
\end{solution}

\vspace{12pt}

\begin{exercise}
  Contoh soal
\end{exercise}

\begin{solution}
  Contoh solusi
\end{solution}

\vspace{12pt}

\begin{exercise}
  Contoh soal
\end{exercise}

\begin{solution}
  Contoh solusi
\end{solution}

\chapter{Analog Transmission}

% Rizky Sandiary 1-4
\begin{exercise}
  Calculate the baud rate for the given bit rate and type of modulation.
  \begin{itemize}
    \item[a.] 2000 bps, FSK
    \item[b.] 4000 bps, ASK
  \end{itemize}
\end{exercise}

\begin{solution}
  We use the formula $S = (1/r) \times N$, but first we need to calculate the value of r for each case.
  \begin{itemize}
    \item[a.] $r = log_22 = 1 \quad \rightarrow \quad S = (1/1) \times (2000 \textnormal{ bps}) = 2000 \textnormal{ baud}$
    \item[b.] 
  \end{itemize}
\end{solution}

\vspace{12pt}

\begin{exercise}
\\
Temukan bandwidth untuk situasi berikut jika kita perlu memodulasi suara 5-KHz.
a. AM
b. PM (set =5)
c. PM (set =1)
\end{exercise}

\begin{solution}
\\
Mengingat frekuensi sinyal suara -

(f)=5kHz(f)=5kHz

a) Bandwidth modulasi amplitudo Bam=2B
AM
=2B
=2 X 5kHz=2×5kHz
=10kHz=10kHz

b) Bandwidth yang dibutuhkan untuk modulasi fase

Bpm=2(1+β)B 
PM
 =2(1+β)B

Sekarang, menggantikan nilai-nilai,

=2(1+3)\times 5 kHz2(1+3)×5kHz
=40kHz=40kHz

c) Bandwidth yang dibutuhkan untuk modulasi fase,

B_{PM}=2(1+\beta)BB 
PM
=2(1+β)B

Sekarang, menggantikan nilai-nilai,
=2(1+1)\times 5kHz=2(1+1)×5kHz
=20kHz=20kHz
\end{solution}

\vspace{12pt}

\begin{exercise}
\\
Saluran telepon memiliki bandwidth 4 KHz. Berapa jumlah bit maksimum yang kami miliki?
dapat mengirim menggunakan masing-masing teknik berikut? Misalkan d = O
sebuah. 
A. ASK
B. QPSK
C. 16-QAM
D. 64-QAM
\end{exercise}

\begin{solution}
Kami menggunakan rumus N = [1/(1 + d)] × r × B, tetapi pertama-tama kita perlu menghitung nilai r untuk setiap kasus. sebuah. 
A. r = log22 = 1 →N= [1/(1 + 0)] × 1 × (4 KHz) = 4 kbps 
B. r = log24=2 →N = [1/(1 + 0)] × 2 × (4 KHz) = 8 kbps 
C. r = log216= 4 →N = [1/(1 + 0)] × 4 × (4 KHz) = 16 kbps 
D. r = log264= 6 →N = [1/(1 + 0)] × 6 × (4 KHz) = 24 kbpsQ19.
\end{solution}

\vspace{12pt}

\begin{exercise}
Sebuah perusahaan kabel menggunakan salah satu saluran TV kabel (dengan bandwidth 6 MHz) untuk menyediakan komunikasi digital bagi setiap penduduk. Berapa kecepatan data yang tersedia untuk setiap penduduk jika perusahaan menggunakan teknik 64-QAM?
\end{exercise}

\begin{solution}
Kita dapat menggunakan rumus: N = [1/(1 + d)] × r × B = 1 × 6 × 6 MHz = 36 Mbps
\end{solution}

\vspace{12pt}

% Dimas Yediberto Luciano Dien 5-8
\begin{exercise}
  Contoh soal 5
\end{exercise}

\begin{solution}
  Contoh solusi
\end{solution}

\vspace{12pt}

\begin{exercise}
  Contoh soal
\end{exercise}

\begin{solution}
  Contoh solusi
\end{solution}

\vspace{12pt}

\begin{exercise}
  Contoh soal
\end{exercise}

\begin{solution}
  Contoh solusi
\end{solution}

\vspace{12pt}

\begin{exercise}
  Contoh soal
\end{exercise}

\begin{solution}
  Contoh solusi
\end{solution}

\vspace{12pt}

% Karel Chavez H 9-12
\begin{exercise}
  Contoh soal 9
\end{exercise}

\begin{solution}
  Contoh solusi
\end{solution}

\vspace{12pt}

\begin{exercise}
  Contoh soal
\end{exercise}

\begin{solution}
  Contoh solusi
\end{solution}

\vspace{12pt}

\begin{exercise}
  Contoh soal
\end{exercise}

\begin{solution}
  Contoh solusi
\end{solution}

\vspace{12pt}

\begin{exercise}
  Contoh soal
\end{exercise}

\begin{solution}
  Contoh solusi
\end{solution}

\chapter{Bandwidth Utilization: Multiplexing and Spreading}

% Muhammad Arie Setya Putra Pala 1-4
\begin{exercise}
\\
Jelaskan tujuan dari multiplexing
\end{exercise}

\begin{solution}
\\
Tujuan multiplexing adalah untuk memungkinkan sinyal ditransmisikan lebih efisien melalui saluran komunikasi tertentu, sehingga mengurangi biaya transmisi.
\end{solution}

\vspace{12pt}

\begin{exercise}
\\
Sebutkan tiga teknik multiplexing utama yang disebutkan dalam bab ini.
\end{exercise}

\begin{solution}
\\
frequency-division multiplexing (FDM), wave-division multiplexing (WDM), and time-division multiplexing (TDM).
\end{solution}

\vspace{12pt}

\begin{exercise}
\\
Bedakan antara tautan dan saluran dalam multiplexing.
\end{exercise}

\begin{solution}
\\
Dalam multiplexing, kata link mengacu pada jalur fisik. Kata saluran mengacu pada bagian dari tautan yang membawa transmisi antara sepasang garis tertentu. Satu tautan dapat memiliki banyak (n) saluran.
\end{solution}

\vspace{12pt}

\begin{exercise}
\\
Manakah dari tiga teknik multiplexing yang digunakan untuk menggabungkan sinyal analog?
Manakah dari tiga teknik multiplexing yang digunakan untuk menggabungkan sinyal digital?
\end{exercise}

\begin{solution}
\\
FDM dan WDM digunakan untuk menggabungkan sinyal analog; bandwidth dibagi. TDM digunakan untuk menggabungkan sinyal digital; waktunya dibagi.
\end{solution}

\vspace{12pt}

% Ricky 5-7
\begin{exercise}
\\
Tentukan hierarki analog yang digunakan oleh perusahaan telepon dan buat daftar level hierarki yang berbeda.
\end{exercise}

\begin{solution}
\\
Hirarki analog menggunakan saluran suara (4 KHz), grup (48 KHz), grup super (240 KHz), grup master (2,4 MHz), dan grup jumbo (15,12 MHz). \\
Struktur analog tertentu menggunakan saluran distribusi kata. (kelas, kelompok, kelas reli, jumbogroup).
\end{solution}

\vspace{12pt}

\begin{exercise}
\\
Tentukan hierarki analog yang digunakan oleh perusahaan telepon dan buat daftar level hierarki yang berbeda.
\end{exercise}

\begin{solution}
\\
Hirarki analog menggunakan saluran suara (4 KHz), grup (48 KHz), grup super (240 KHz), grup master (2,4 MHz), dan grup jumbo (15,12 MHz). \\
Struktur analog tertentu menggunakan saluran distribusi kata. (kelas, kelompok, kelas reli, jumbogroup).
\end{solution}

\vspace{12pt}

\begin{exercise}
\\
Manakah dari tiga teknik multiplexing yang umum untuk link serat optik? Jelaskan alasannya.
\end{exercise}

\begin{solution}
\\
WDM umum untuk multiplexing sinyal optik karena memungkinkan multiplexing sinyal dengan frekuensi yang sangat tinggi.
\end{solution}

\vspace{12pt}

% Fajar Bimantara 8-11
\begin{exercise}
  Contoh soal 8
\end{exercise}

\begin{solution}
  Contoh solusi
\end{solution}

\vspace{12pt}

\begin{exercise}
  Contoh soal 9
\end{exercise}

\begin{solution}
  Contoh solusi
\end{solution}

\vspace{12pt}

\begin{exercise}
  Contoh soal
\end{exercise}

\begin{solution}
  Contoh solusi
\end{solution}

\vspace{12pt}

\begin{exercise}
  Contoh soal
\end{exercise}

\begin{solution}
  Contoh solusi
\end{solution}

\vspace{12pt}

% Julicko Pratama Putra 12-14
\begin{exercise}
  Contoh soal 12
\end{exercise}

\begin{solution}
  Contoh solusi
\end{solution}

\vspace{12pt}

\begin{exercise}
  Contoh soal
\end{exercise}

\begin{solution}
  Contoh solusi
\end{solution}

\vspace{12pt}

\begin{exercise}
  Contoh soal
\end{exercise}

\begin{solution}
  Contoh solusi
\end{solution}

\vspace{12pt}

% Nadjamudin Beda 15-18
\begin{exercise}
  Contoh soal 15
\end{exercise}

\begin{solution}
  Contoh solusi
\end{solution}

\vspace{12pt}

\begin{exercise}
  Contoh soal
\end{exercise}

\begin{solution}
  Contoh solusi
\end{solution}

\vspace{12pt}

\begin{exercise}
  Contoh soal
\end{exercise}

\begin{solution}
  Contoh solusi
\end{solution}

\vspace{12pt}

\begin{exercise}
  Contoh soal
\end{exercise}

\begin{solution}
  Contoh solusi
\end{solution}

\end{document}

